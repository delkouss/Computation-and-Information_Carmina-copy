\chapter*{Foreword}
\addcontentsline{toc}{chapter}{\textcolor{ocre}{Foreword}} % Add a Bibliography heading to the table of contents
We are in the middle of the so-called informatoin society. Most of our interactions are mediated in one or another way by communication and computation devices which also majoritarily are connected to the internet. This information society facilitates many of our tasks, from looking for a neearby supermarket to learning about the best treatment for whatever illness we have or think we have (don't). It guides our choice of job, our application for job positions and allows us to communicate with distant relatives and also with friends that might be a few meters away from us. All this wealth of possibilities capable of satisfying our pst needs and drivign force of needs we did ot have a few years ago, relies on mathematical tools and models that did not exist until very recently. In fact some of these tools were thought to be impossible until they wre proved wrong. Even while part of the information society, we are also in the middle of a revolution, that is taking us from the world of classical information to a new world were information follows the strange laws of quantum mechanics. We expect much of this new world, amazing speed ups in computation, that will allow to discover new drugs, produce unhackable communication systems or improve our observations of the distant universe. However, what does it mean to speed up, what are the limits of the classical world? In which sense are some problems untractable by the computers we know today? Sure, it is only a matter of buying enough computation power and then we could also achieve the feats mentioned before, or perhaps is there some catch? It turns out, that there are problems that no computation device can solve, problems that can be computed but are beyond the reach of quantum computers. In fact, most of the problems that classical computers can deal with, can also be dealt with by quantum computers with no speed up. It is only for a limited number, of highly structured problems that quantum ocmputers promise to offer amazing improvement. However, still for those problems the speed up is in terms of how the computation resources necessary grow with the size of the problem. What is even information?


