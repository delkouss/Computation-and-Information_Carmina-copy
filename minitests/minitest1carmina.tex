\noindent \textbf{Problem 2:} \emph{Coding information} (40 points total)\\
This  assembly language instruction $add$  $\$t0,\$t3, \$t5$  performs an addition between two operands.
Its corresponding R-type instruction has the following field values: 

\begin{table}[h!]
\begin{tabular}{llllcl}
\hline
\multicolumn{1}{|c|}{0} & \multicolumn{1}{c|}{11} & \multicolumn{1}{c|}{13} & \multicolumn{1}{c|}{8} & \multicolumn{1}{c|}{0} & \multicolumn{1}{c|}{32} \\ \hline
6 bits                  & 5 bits                  & 5 bits                  & 5 bits                 & 5 bits                 & 6 bits                 
\end{tabular}
\end{table}


\begin{enumerate}

\item Convert this instruction into machine code (hexadecimal). (10 points) %more precise

\noindent\fbox{
  \parbox{.9\textwidth}{
Answer 2a\hfill
    \vspace{8cm}
  }
}

\item Take the four most significant digits of the previous hexadecimal number and convert the resulting 4-digit hexadecimal number to decimal. (5 points) 
(Use $015F_{16}$ in case you did not obtain the hexadecimal number in question 2a)

\noindent\fbox{
  \parbox{.9\textwidth}{
Answer 2b\hfill
    \vspace{4cm}
  }
}

\newpage
\item Let's assume now that the values stored in registers $\$t3$ and $\$t5$ are the following 6-bit two's complement numbers: 110110 and 100011. What is the decimal number they represent?  (5 points) 

\noindent\fbox{
  \parbox{.9\textwidth}{
Answer 2c\hfill
    \vspace{6cm}
  }
}


\item Convert the previous 6-bit numbers to decimal assuming now they are 6-bit sign/magnitude numbers (5 points). 

\noindent\fbox{
  \parbox{.9\textwidth}{
Answer 2d\hfill
    \vspace{6cm}
  }
}

\newpage
\item Perform the addition of the two 6-bit two's complement numbers of question 2c (assume the result is also a 6-bit two's complement number). Is the result correct? If not, explain why. (5 points) 

\noindent\fbox{
  \parbox{.9\textwidth}{
Answer 2e\hfill
    \vspace{6cm}
  }
}

\item Convert the previous 6-bit two's complement numbers to 8-bit two's complement numbers.  (10 points)

\noindent\fbox{
  \parbox{.9\textwidth}{
Answer 2f\hfill
    \vspace{6cm}
  }
}





%\item Bonus question
\end{enumerate}

\newpage

\noindent \textbf{Problem 3:} \emph{Representation of switching functions} (10 points total)\\

Consider the the following switching expression: $f(x_1,x_2,x_3)=x_{1}^{'}x_{2} + x_{2}^{'}x_{3} + x_{1}x_{2}x_{3}^{'} $

\begin{enumerate}

\item Determine its truth table (8 points) %more precise

\noindent\fbox{
  \parbox{.9\textwidth}{
Answer 3a\hfill
    \vspace{6cm}
  }
}

\item What is its 1-set? (2 points) %more precise

\noindent\fbox{
  \parbox{.9\textwidth}{
Answer 2b\hfill
    \vspace{6cm}
  }
}


\end{enumerate}

