\noindent \textbf{Problem 1:} \emph{Parity check matrix} (25 points total)\\
Consider the following parity check matrix:
\begin{equation*}
\begin{pmatrix}
0 & 0 & 0 & 1 & 1 & 1\\
0 & 1 & 1 & 0 & 0 & 1\\
1 & 0 & 1 & 0 & 0 & 0\\
\end{pmatrix}
\end{equation*}

\begin{enumerate}
\item (5 points) Does the matrix correspond to a Hamming code? Explain why yes or no.

\noindent\fbox{
  \parbox{.9\textwidth}{
\hfill
    \qu{\vspace{3cm}}

    \soln{}
  }
}

\item (15 points) Find the minimum distance of a code with the previous parity check matrix.

\noindent \fbox{
  \parbox{.9\textwidth}{
\hfill
    \qu{\vspace{13cm}}

    \soln{}
  }
}

\item (5 points) Suppose that you receive the word $y=(0011001)$, is $y$ a codeword of the code with the previous parity check matrix?

\noindent\fbox{
  \parbox{.9\textwidth}{
\hfill
    \qu{\vspace{10cm}}

    \soln{}
  }
}

\end{enumerate}

\noindent \textbf{Problem 2:} \emph{Channel capacity} (25 points total)\\
Consider a discrete memoryless channel with input alphabet of size four and output alphabet of size six. The transition matrix is given by
\begin{equation*}
p(y|x)=\begin{pmatrix}
1-e & e & 0 & 0 &0&0\\
0 & e & 1-e & 0 &0&0\\
0 & 0 &0  & 1-e & e&0\\
0 & 0 &0  & 0 & e &1-e
\end{pmatrix}
\end{equation*}
where $p\in[0,1]$.
\begin{enumerate}
\item (5 points) Does the transition matrix above correspond to a weakly symmetric channel? Indicate why or why not.

\noindent\fbox{
  \parbox{.9\textwidth}{
\hfill
    \qu{\vspace{3cm}}

    \soln{}
  }
}
\item (20 points) What is the capacity of the channel in bits? (We will give partial credit if you can find non-trivial bounds on the value)

\noindent\fbox{
  \parbox{.9\textwidth}{
\hfill
    \qu{\vspace{19cm}}

    \soln{}
  }
}
\end{enumerate}
