\noindent \textbf{Problem 1:} \emph{Data compression} (50 points total)\\
\begin{enumerate}
\item (5 points) Is the code $C=\{0,01,011,0111\}$ instantaneous? Indicate why or why not.

\noindent\fbox{
  \parbox{.9\textwidth}{
\hfill
    \qu{\vspace{3cm}}

    \soln{}
  }
}
%\item (5 points) Is the code $C=\{1,10,0\}$ a prefix code? Indicate why or why not.
\item (10 points) Find whether or not the code $C=\{01,100,1101,10111,01011\}$ is uniquely decodable via the algorithm of Sardinas-Patterson.

\noindent\fbox{
  \parbox{.9\textwidth}{
\hfill
    \qu{\vspace{10cm}}

    \soln{}
  }
}\item (5 points) Can the code $C=\{0,10,110,1110\}$ be a Huffman code? Indicate why or why not.

\noindent\fbox{
  \parbox{.9\textwidth}{
\hfill
    \qu{\vspace{2cm}}

    \soln{}
  }
}\item (10 points) Can a uniquely decodable binary code with lengths $\{1,2,3,3,3\}$ exist? Indicate why or why not.

\noindent\fbox{
  \parbox{.9\textwidth}{
\hfill
    \qu{\vspace{5cm}}

    \soln{}
  }
}\item (10 points) Construct the Huffman code of an ensemble with symbols $\{a,b,c,d\}$ that occur with probabilities $\{0.35,0.3,0.25,0.1\}$.

\noindent\fbox{
  \parbox{.9\textwidth}{
\hfill
    \qu{\vspace{13cm}}

    \soln{}
  }
}\item (5 points) What is the average length of the Huffman code in the previous exercise? If you were not able to find it, assume that the codewords for the symbols $\{a,b,c,d\}$ are respectively $\{0,10,110,111\}$.

\noindent\fbox{
  \parbox{.9\textwidth}{
\hfill
    \qu{\vspace{4.5cm}}

    \soln{}
  }
}\item (5 points) Let $X$ be an ensemble with alphabet $\{a,b,c\}$ and probabilities $\{0.8,0.1,0.1\}$. Construct the extended ensemble $X^2$. That is: 1) specify the alphabet of $X^2$ and 2) give the probabilities of each symbol in the alphabet of $X^2$.

\noindent\fbox{
  \parbox{.9\textwidth}{
\hfill
    \qu{\vspace{12cm}}

    \soln{}
  }
}
%\item[] (10 points, bonus) Let us consider the opposite goal to our previous efforts. Suppose that you want to transform a sequence of uniform binary random variables $X_1,X_2,\ldots$ into some random variable $Y$:
%\begin{itemize}
%\item Describe a general procedure that allows to produce samples of $Y$ from outcomes of uniform binary random variables.
%\item Bound the average number of samples necessary to produce a sample of $Y$ is $Y$ is a binary random variable with probabilities $\{p,1-p\}.$
%\end{itemize}
%\noindent\fbox{
%  \parbox{.9\textwidth}{
%Answer bonus\hfill
%    \qu{\vspace{12cm}}
%    \soln{}
%  }
%}
\end{enumerate}



